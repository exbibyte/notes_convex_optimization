\documentclass[12pt,letter]{article}

%% \usepackage[fleqn]{amsmath}
\usepackage[margin=1in]{geometry}
\usepackage{amsmath,amsfonts,amsthm,bm}
\usepackage{breqn}
\usepackage{amsmath}
\usepackage{amssymb}
\usepackage{tikz}
\usepackage{algorithm2e}
\usepackage{siunitx}
\usepackage{graphicx}
\usepackage{subcaption}
%% \usepackage{datetime}
\usepackage{multirow}
\usepackage{multicol}
\usepackage{mathrsfs}
\usepackage{fancyhdr}
\usepackage{fancyvrb}
\usepackage{parskip} %turns off paragraph indent
\pagestyle{fancy}

\usetikzlibrary{arrows}

\DeclareMathOperator*{\argmin}{argmin}
\newcommand*{\argminl}{\argmin\limits}

\newcommand{\mathleft}{\@fleqntrue\@mathmargin0pt}
\newcommand{\R}{\mathbb{R}}
\newcommand{\Z}{\mathbb{Z}} 
\newcommand{\N}{\mathbb{N}}
\newcommand{\norm}[1]{\|#1\|}

\setcounter{MaxMatrixCols}{20}

% remove excess vertical space for align equations
\setlength{\abovedisplayskip}{0pt}
\setlength{\belowdisplayskip}{0pt}
\setlength{\abovedisplayshortskip}{0pt}
\setlength{\belowdisplayshortskip}{0pt}

\begin {document}

\lhead{Notes - Convex Optimization, 2020/01/17}

\begin{enumerate}
\item Polyhedral
  \begin{align*}
    P=\{x: Ax \leq b, Cx=D \}
  \end{align*}

\item Euclidean Ball
  \begin{align*}
    B(x_c,x) = \{ x: \norm{x-x_c}_2 \leq r \}
  \end{align*}
  Can use affine combination and triangular inequality of norm to prove convexity.
\item Elipse
  \begin{align*}
    E(x_c,P) = \{ x: (x-x_c)^T P^{-1} (x-x_c) \leq 1 \}, P > 0\\
    P=r^2 I \implies \text{ Euclidean Ball }\\
    P=Q \begin{bmatrix}
      \lambda_1 & ..\\
      .. & \lambda_n\\
    \end{bmatrix}
    Q^T\\
    (x-x_c)^T Q \begin{bmatrix}
      \lambda_1 & ..\\
      .. & \lambda_n\\
    \end{bmatrix}
    Q^T(x-x_c) \leq 1\\
    \tilde{x}^T \begin{bmatrix}
      \lambda_1 & ..\\
      .. & \lambda_n\\
    \end{bmatrix}
    \tilde{x}^T \leq 1\\
    \tilde{x}^T \begin{bmatrix}
      \frac{1}{\lambda_1} & ..\\
      .. & \frac{1}{\lambda_n}\\
    \end{bmatrix}
    \tilde{x} = \frac{\tilde{x_1}^2}{\lambda_1}+..+\frac{\tilde{x_n}^2}{\lambda_n}\leq 1\\
    \text{volum of elipsoid proportional to } \sqrt{det(P)}=\sqrt{\Pi_i \lambda_i}
  \end{align*}
  \item Norm Ball
  \begin{align*}
    C=\{x: \norm{x} \leq 1 \} \text{ for any } \norm{*}
  \end{align*}
  Example: lp-norm\\
  
\item Cone
  \begin{align*}
    (\forall x \in C, \forall \theta \geq 0) \theta x \in C\\
  \end{align*}
\item Convex Cone
  Eg: $S^n, S^N_+$ are convex cones\\
  convexity check for $S^N_+$:\\
  \begin{align*}
    x_1 \in S^n_+ \implies v^Tx_1v \geq 0\\
    x_2 \in S^n_+ \implies v^Tx_2v \geq 0\\
    v^T(\theta x_1 + (1-\theta)x_2)v \geq 0\\
    v^T\theta x_1 v + (1-\theta)v^T x_2 v \implies x \in S^n_+\\
  \end{align*}
  convexity check for cone:\\
  \begin{align*}
    x_1 \in S^n_+ \implies \theta x_1 \in S^n_+, \theta \geq 0\\
    (\forall v) v^T x v \geq 0 \implies v^T(\theta x) v \geq 0 \implies \text{ cone}
  \end{align*}
\item Proper Cone\\
  Definition:
  \begin{itemize}
  \item convex
  \item closed(contains all boundary points)
  \item solid(non-empty interior)
  \item pointed(contains no line): $x \in K \implies -x \in K$
  \end{itemize}
  Then the proper cone K fefines a generalized inequality ($\leq_K$) in $\R^n$
  \begin{align*}
    x \leq_K y \implies y-x \in K\\
    x <_K y \implies y-x \in int(K)
  \end{align*}
  Example: $K=R^n_+$ (non-negative orthant):
  \begin{align*}
    n = 2\\
    x \leq_{R^2_+} y \implies y-x \in R^2_+
  \end{align*}
  Cone provides partial ordering using difference of 2 objects
  Example: $X\ leq_{S^n_+} Y \implies Y-X \in S^n_+$ (Y-X is PSD)

  \pagebreak
  
\item Operations Preserving Convexity
  \begin{itemize}
  \item intersection
    \begin{align*}
      S{\alpha}\text{ is affine, convex, convex cone} \forall \alpha \in A\\
      \cap_{\alpha \in A} S_{\alpha} \text{ is affine, convex, convex cone}
    \end{align*}
    Example:Polyhedral is the intersection of some halfspaces and hyperplanes, so it is convex.\\
    Any cloed convex set can be represented by possibly infinitely many half spaces.
  \item affine functions\\
    let $f(x)=Ax+b, f:\R^n \to \R^m$\\
    then if S is a convex set we have:
    \begin{itemize}
    \item project forward: $f(S) = \{ f(X) : X \in S \}$ is convex
    \item project back: $f^{-1}(S) = \{ X : f(X) \in S \}$ is convex
    \end{itemize}
    Example:
    \begin{align*}
      C = \{ y : y=Ax+b, \norm{x} \leq 1\}
    \end{align*}
    $\norm{x} \leq 1$ is convex, $Ax+b$ is affine $\implies$ C is convex\\
    Example:
    \begin{align*}
      C = \{ x : \norm{Ax+b} \leq 1 \}
    \end{align*}
    $\{y: \norm{y} \leq 1 \}$\\
    $y$ is an affine function of $x \implies$ C is convex\\
  \end{itemize}
\item Affine Functions
  $A(x)=\sum_i x_i A_i + B, A_i\in S^n, B \in S^n, x\in \R^n$\\
  is $\{X : \sum_i x_i A_i = \tilde{A}(x)\leq B \}$ convex?\\
  let $y = B-\tilde{A}(x)$\\
  $\{y: 0 \leq y \}$ is convex\\
  We know $\{y: 0 \leq y \}$ is convex. Further y is an affine function of x $\implies$\\
  $\{X : \sum_i x_i A_i \leq B \}$ is also convex\\
  $\{X : \sum_i x_i A_i \leq B \}$ is a Linear Matrix Inequality

  \pagebreak
  
\item Properties of convex sets
  \begin{itemize}
  \item Separating Hyperplanes: if $S,T \subset \R^n$ are convex and disjoint, then $\exists a \neq 0, b$ such that:\\
    \begin{align*}
      a^Tx \geq b, forall x \in S\\
      a^Tx \leq b, forall x \in T\\
    \end{align*}
  \end{itemize}
\item Supporting Hyperplane:\\
  if $S$ is convex, $\forall x_0 \in \partial S$ (boundary of S), then $\exists a \neq 0$ such that $a^Tx \leq a^Tx_0, \forallx \in S$\\
\end{enumerate}

\end{document}

  